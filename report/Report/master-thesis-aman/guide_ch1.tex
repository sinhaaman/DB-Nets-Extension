\chapter{Introduction}

\paragraph*{\textnormal{In today's world, there is an increasing demand to integrate master data and business process. Traditionally, there has been isolation between data management and business processes. This isolation, in the organisational structure, might lead to fragmentation and redundancy as there are few experts and tools which focus only on master data, and few focus only on processes.}}

\subparagraph*{\textnormal{As per \cite{DBLP:journals/corr/DBNets}, the BP management systems (BPMSs), such as Bizagi BPM, Bonita BPM, Activiti, Camunda, and YAWL, provide conceptualization for process control flow along with some joins between the control flow and data as:
\begin{itemize}
	\item local data is attached with the process instances.
	\item using a database to store persistence data.
	\item while choosing a path among multiple alternatives, the decision logic tries to query the persistent data.
	\item the task logic specifies how to update local and
	persistent data.
\end{itemize}
Still there is no well established approach to express the decision and task logic. Even if it exists, it is handled in an ad-hoc way by combining tool specific language with general purpose programming language such as JAVA. As a result, the interaction between processes and data is exploited at the time of process enactment and is not conceptually well understood. This leads to inaccuracy in performing verification tasks.}}

\subparagraph*{\textnormal{Foundational research centred on either data management, e.g. database theory, or process control, e.g. Petri nets, also faced the similar issue of data integration. Attempts have been made to represent the business process in a formal way using Coloured Petri nets(CPNs) where the colours account for the data types and the tokens carry the data value through the net. In this approach, the verification task was tackled by restricting the domain of the colours to a finite set, hence predetermining the way tokens carry data. In all such approaches, the data was locally attached with the control flow instead of providing support for global, persistent relational data. Using database theory, conceptual modelling and formal methods, an attempt to data-aware processes, under data-centric approaches, emerged. In all such approaches, the processes are centred around persistence data, which maintains the relevant information about the domain of interest, along with capturing semantics in terms of classes, relations and constraints. The evolution of master data, using CRUD(Create-Read-Update-Delete) operations, can be modelled by performing atomic tasks over the data components. In data centric process models, there is an implicit representation for the control flow, however, they disregard explicit representation of sequencing tasks over time.}}

\subparagraph*{\textnormal{An attempt, to integrate master data and business processes, is made by Rivkin and Montali in \cite{DBLP:journals/corr/DBNets} where they present DB-nets, a relational extension of Colored Petri nets, which could integrate persistence data, modelled in a relational database, along with process control flow, modelled using Colored Petri nets. In DB-nets, the interaction between the master data and business processes is handled by a separate layer, for which Rivkin and Montali provide a theory in their paper. Modelling and execution semantics for DB-nets are provided along with the possibility to perform analysis over them.}}

\subparagraph*{\textnormal{In this thesis, we present a light weight theory for DB-nets which is based on \cite{DBLP:journals/corr/DBNets}. Later, we develop an extension in CPN Tools \cite{CPN_Tools}, a tool to model Coloured Petri nets, which facilitates modelling and execution of DB-nets. In the end, we mention our approach for performing verification of DB-nets. We provide preliminaries in order to get the reader acquainted with the basics required to understand the thesis. Here, we assume the reader is familiar with the syntax, semantics and verification of Petri nets \cite{DBLP:books/daglib/Reisig2013}. In the appendix (see \ref{ch:app_comp_tools}), we justify our decision to select CPN Tools for developing the extension.}}

\begin{comment}
\paragraph*{\textnormal{Modelling dynamic systems is a challenge in today's world. The reason for the same is that due to their complexity and the property of concurrency and non-determinism, their execution path can be numerous. One simply cannot build the concurrent systems without proper analysis. The challenge is to build an executable model of the system. This will help in analysing systems and simulating them without actually building the real system. There are many application domains of CPNs which includes communication protocols, data networks, distributed algorithms, embedded systems etc.}}

\section{Introduction}
\paragraph*{\textnormal{In today's world, there is an increasing demand to integrate master data and business process. By business processes, we mean the processes which are carried out keeping business in mind. For example, in chapter \ref{ch:CPN} we considered a process for booking a taxi. By master data we mean the data which takes part or is used in a defined business process. For example, in the previous chapter we considered the CPN model for taxi booking. In the net (figure \ref{fig:CPN_Taxi_Booking}), the data which takes part in the CPN model is provided at the places. In this case, we set the data values at the places ourselves, but in reality we acquire this data at the runtime of the process. For example, the CPN provided in the last chapter (figure \ref{fig:CPN_Taxi_Booking}), a user might use a webpage to provide the pickup data and the phone number which might be arbitrary whereas in the model the data values are fixed.}}



\subparagraph*{\textnormal{In order to model, verify, mine and perform other activities on data-aware business process, an equilibrium must be established between data and process related aspects. Nowadays, the data is limited but the data components are huge. In this chapter, in order to model the data related aspect we use relational databases and in order to model the process related aspect we use the theory of CPN. A formalism called db-nets is introduced in the next section. In order to read about relational databases refer {\color{red} appendix}.}}

\paragraph*{\textnormal{Coloured Petri Nets are Petri Nets which differ in the token structure.
Formally, Coloured Petri Nets(CPNs) are petri nets having nine tuples. The formal definition is discussed later in this section. According to Jensen in \cite{Jensen2007}, informally, CPN is a graphical modelling language combining Petri nets along with a high level programming language. Jensen uses standard ML as the high level programming language. However, we are not providing the syntax or semantics of ML for defining colour sets, declaring variables, and specifying initial markings, guards, and arc expressions in CPN models. Here we just define the concrete part of Petri net and instead of CPN-ML, one can replace it with any other programming language. Also, we assume that the programming language provides functionality to evaluate the expression and their types, also it provides constructs for defining data types and declaring variables. Here there are two components of the booking system. One is the data, whereas the other is the workflow of booking the taxi. In this chapter, the data part is assumed to be fixed and we provide the data in our model.}}

\end{comment}


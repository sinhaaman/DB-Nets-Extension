\chapter{Database Theory (Basics)}
\label{ch:db_basic}
\paragraph*{\textnormal{This chapter talks about the basic theory behind the database. We start with concepts in first order logic(FOL), and then we discuss first order logic queries followed by the discussion of translation of queries. For understanding the basic concepts of databases in this chapter, we refer \cite{DBLP:books/aw/found_DB}. \todo{citing diego slides?} Here we would discuss briefly about the FOL with equality which we will use in FOL queries.}}

\section{First Order Logic (FOL)}
\label{sec:db_theory_FOL}
\paragraph*{\textnormal{This logic can be used to represent the objects of the domain of discourse(the universe), about the properties of the objects and their relationships\footnote{properties corresponds to the unary predicate and the relations corresponds to \textit{n}-ary predicate}. Functions and Constants are also included in first order logic.}}

\subsection*{Syntax}
\subparagraph*{\textnormal{Variables $x_{1},x_{2},\ldots,x_{n}$ are called individual variables which denote single objects. The set of variables is denoted by:
\begin{equation*}
Vars = \{x_{1},x_{2},\ldots,x_{n}\}
\end{equation*}
A function symbol of arity $k$, where $k \geq 0$ and $x_{1},x_{2},\ldots,x_{k}$ are variables, is denoted as:
\begin{equation*}
f^{k}(x_{1},x_{2},\ldots,x_{k})
\end{equation*}
}}

\begin{defs}
The set of \textbf{Terms} is defined inductively as smallest possible set satisfying:
\begin{enumerate}
\item Each variable is a term . i.e. Vars $\subseteq$ Terms;
\item If $t_{1},t_{2},\ldots,t_{k} \in Terms$ and $f^{k}$ is a k-ary function symbol, then $f^k(t_{1},t_{2},\ldots,t_{k}) \in Terms$.
\end{enumerate}	 
\end{defs}

\begin{defs}
	The set of \textbf{Formulas} is defined inductively as smallest possible set satisfying:
	\begin{enumerate}
		\item If $t_{1},t_{2},\ldots,t_{k} \in Terms$ and $P^{k}$ is a k-ary predicate \footnote{a predicate of arity 0 is a proposition}, then $P^{k}(t_{1},t_{2},\ldots,t_{k}) \in$ Formulas(atomic formulas).
		\item If $t_{1},t_{2} \in Terms$, then $t_{1} = t_{2} \in Formulas$. (equality)
		\item If $\varphi \in Formulas$ and $\varphi \in Formulas$ then
		\begin{itemize}
			\item $\neg\varphi \in Formulas$
			\item $\varphi \wedge \psi \in Formulas $
			\item $\varphi \wedge \psi \in Formulas $
			\item $\varphi \rightarrow \psi \in Formulas$
		\end{itemize}
		\item If $\varphi \in Formulas$ and $x \in Vars$ then
		\begin{itemize}
			\item $\exists x.\varphi \in Formulas$
			\item $\forall x.\varphi \in Formulas$
		\end{itemize} 
	\end{enumerate}	 
\end{defs}

\subsection*{Semantics}
\paragraph*{\textnormal{Following the FOL syntax, we will have a look at the semantics of the language.}}
\begin{defs}
	Given an \textbf{alphabet} of predicates $P_{1},P_{2},\dots$ and function symbols $f_{1},f_{2},\ldots$ each with an associated arity, a FOL \textbf{interpretation} is:
	\begin{equation*}
		I = (\Delta^{I},\cdot^{I})
	\end{equation*}
where:
\begin{itemize}
	\item $\Delta^{I}$ is the interpretation domain (a non-empty set of objects);
	\item $\cdot^{I}$ is the interpretation function that interprets predicates and function symbols as:
	\begin{itemize}
		\item if $P_{i}$ is a k-ary predicate, then $P_{i}^{I} \subseteq \Delta^{I} \times \ldots \times \Delta^{I}$ (k times)
		\item if $f_{i}$ is a k-ary function, $k \geq 1$, then $f_{i}^{I} : \Delta^{I} \times \ldots \times \Delta^{I} \rightarrow \Delta^{I}$
		\item if $f_{i}$ is a constant, then $f_{i}^{I} : () \rightarrow \Delta^{I}$, which means $f_{i}$ denotes exactly one object in the domain.
	\end{itemize} 
\end{itemize}
\end{defs}

\begin{defs}
	Given an interpretation I, an \textbf{assignment} is a function
	\begin{equation*}
		\alpha : Vars \rightarrow \Delta^{I}
	\end{equation*}
that assigns to each variable $x \in Vars$ an object $\alpha(x) \in \Delta^{I}$. We could extend the notion of assignments to terms. We define a function $\hat{\alpha} : Terms \rightarrow \Delta^{I}$ inductively as follows:
\begin{itemize}
	\item $\hat{\alpha}(x) = \alpha (x)$, if $x \in Vars$
	\item $\hat{\alpha}(f(t_{1},\ldots,t_{k})) = f^{I}(\hat{\alpha}(t_{1}),\ldots,\hat{\alpha}(t_{k}))$
	\item for constants $\hat{\alpha}(c) = c^{I}$
\end{itemize}
\end{defs}

\subparagraph*{\textnormal{We define a FOL formula $\varphi$ is true in a interpretation $I$ with respect to an assignment $\alpha$, written $I,\alpha \models \varphi$:}}
\begin{equation*}
\begin{cases}
I,\alpha \models \varphi & if $(\hat{\alpha}(t_{1}),\ldots,\hat{\alpha}(t_{k})) \in P^{I}$\\
I,\alpha \models t_{1} = t_{2} &  if $\hat{\alpha}(t_{1}) = \hat{\alpha}(t_{2})$\\
I,\alpha \models \neg \varphi & if $I,\alpha \nvDash \varphi$ \\
I,\alpha \models \varphi \wedge \psi & if $I,\alpha \models \varphi$ and $I,\alpha \models \psi$\\
I,\alpha \models \varphi \vee \psi & if $I,\alpha \models \varphi$ or $I,\alpha \models \psi$\\
I,\alpha \models \varphi \rightarrow \psi & if $I,\alpha \models \varphi$ implies $I,\alpha \models \psi$\\
I,\alpha \models \exists x .\varphi & if for some $a \in \Delta^{I}$ we have $I,\alpha\lbrack x \mapsto a \rbrack \models \varphi$\\
I,\alpha \models \forall x .\varphi & if for every $a \in \Delta^{I}$ we have $I,\alpha\lbrack x \mapsto a \rbrack \models \varphi$
\end{cases}
\end{equation*}

\subparagraph*{\textnormal{where $\alpha\lbrack x \mapsto a \rbrack$ stands for the new assignment obtained from $\alpha$ as follows:}}
\begin{equation*}
\begin{aligned}
\alpha\lbrack x \mapsto a \rbrack(x) &= a \\
\alpha\lbrack x \mapsto a \rbrack(y) &= \alpha (y),\hspace*{1 cm} \textnormal{for }y \neq x
\end{aligned}
\end{equation*}

\subsection*{Relational Schema and Database Schema}
\subparagraph*{\textnormal{Inspired from \cite{DBLP:books/aw/found_DB}, we assume \textit{domain} as a countably infinite set $\Delta$, \textit{attributes} as a finite set \textit{U}, and a mapping function $dom:U\rightarrow\Delta$ and $dom(A)$ is called domain of $A$, where $A \in U$.}}

\begin{defs}
	\label{defs:dbn_sort_function}
	A function \textit{sort} is defined as $sort:R^{n}\rightarrow{\mathcal{P}^{fin}}(U)$, where ${\mathcal{P}^{fin}}$ is the finitary powerset\footnote{the set of finite subsets} of attributes, $R^{n}$ is the set of relation schema and $U$ is the finite set of attributes.
\end{defs}

\subparagraph*{\textnormal{The sort of a relation schema \textit{R} is simply written as \textit{sort(R)} and the arity is written as $arity(R) = |\textit{sort(R)}|$. A relational schema (\textit{R}) and set of attributes(\textit{U}) together make up the structure of the table. $R[U]$ is used to represent $sort(R) = U$, and $R[n]$ to represent $arity(R) = n$.}}

\begin{defs}
	\label{defs:dbn_database_schema}
	For a given domain $\Delta$, a \textbf{database schema} is a non empty finite set $\mathcal{R}$ of relational schemas, written as:
	\begin{equation*}
	\mathcal{R} = \{R_{1}[U_{1}],\ldots,R_{n}[U_{n}]\}
	\end{equation*}
	where $R_{1},\ldots,R_{n}$ are relational schemas and $U_{1},\ldots,U_{n}$ are attributes
\end{defs}

\subsection*{First order logic queries}
\begin{defs}
A first order logic \textbf{query} is an (open) FOL formula. Let $\varphi$ be a first order logic query with free variables $(x_{1},\ldots,x_{k})$, the query is written as:
\begin{equation*}
\varphi (x_{1},\ldots,x_{k})
\end{equation*}
and we say that the query has arity k.
\end{defs}
\paragraph*{\textnormal{For a given interpretation $I$, the only interesting assignments are those which map the variables $x_{1},\ldots,x_{k}$. We write an assignment $\alpha$ such that $\alpha(x_{i}) = a_{i},$ for $i = 1,\ldots,k$ as $\langle a_{1},\ldots,a_{k} \rangle$.}}

\begin{defs}
Given an interpretation $I$, the \textbf{answer to a query} $\varphi (x_{1},\ldots,x_{k})$ is :
\begin{equation*}
{\varphi (x_{1},\ldots,x_{k})}^{I} = \{(a_{1},\ldots,a_{k}) | I,\langle a_{1},\ldots,a_{k} \rangle \models \varphi (x_{1},\ldots,x_{k}) \}
\end{equation*}
\end{defs}

\subparagraph*{\textnormal{Notation $\varphi ^ I$ is used which keeps the free variables implicit, and $\varphi(I)$ is used for making apparent that $\varphi$ becomes a function from interpretations to set of tuples.}}

\begin{defs}
A \textbf{first order logic boolean query} is a first order logic query without free variables.
\end{defs}

\subparagraph*{\textnormal{The answer to the boolean query is defined as:
\begin{equation*}
{\varphi()}^{I} = \{() | I,\langle \rangle \models \varphi() \}
\end{equation*}
such an answer is :
\begin{itemize}
\item \bdq{true}, the empty tuple (), if $I \models \varphi$
\item \bdq{false}, the empty set $\emptyset$, if $I \nvDash \varphi$
\end{itemize}}}

\subsection*{\color{red} Translation of queries needs to be written.}

\subparagraph*{\textnormal{A \textit{\textbf{multiset} m} over a non-empty set $S$ can be thought as a function which maps an element $s \in S$ to a natural number in $\mathbb{N}$ which denotes the number of appearances of the element $s$. The natural number $m(s)$ is also called coefficient of $s$ in $m$. We use ${}^{\backprime}$ as the infix operator. The number preceding the symbol ${}^{\backprime}$ signifies the coefficient of $s$ in $m$ while the tuple succeeding the symbol represents the element.
}}
\subparagraph*{\textnormal{Let us formally define multisets:}}
\begin{defs}
	\label{defs:Multiset}
	Let $S = \{ s_{1}, s_{2}, s_{3}, . . . \}$ be a non-empty set. A \textbf{multiset} over S is
	a function $m : S \rightarrow \mathbb{N}$ that maps each element $s \in S$ into a non-negative integer $m(s) \in \mathbb{N}$ called the \textbf{number of appearances} (coefficient) of $s, where s \in S$, in $m$. A multiset $m$ can also be written as a sum:
	\begin{equation*}
	_{}^{++}\sum\limits_{s \in S}^{ } {m(s)}^{\backprime}s = {m(s_{1})}^{\backprime} s_{1} +\!\!+ \ {m(s_{2})}^{\backprime}s_{2} +\!\!+ \ {m(s_{3})}^{\backprime} s_{3} +\!\!+ \ldots
	\end{equation*}
	\textbf{Membership, addition, scalar multiplication, comparison,} and \textbf{size} are defined
	as follows, where $m_{1}, m_{2}$, and m are multisets, and $n \in \mathbb{N}$ :
	\begin{enumerate}
		\item $\forall s \in S$ : $s \in m \Leftrightarrow  m(s) > 0$. (Membership)
		\item $\forall s \in S$ : $(m_{1}$ $+\!+$ $m_{2})(s)$ = $m_{1}(s) + m_{2}(s)$. (Addition)
		\item $\forall s \in S : (n$ $\ast\!\ast$ $m)(s) = n$ $\ast$ $m(s)$. (Scalar multiplication)
		\item $m_{1}$ $\ll=$ $m_{2}$ $\Leftrightarrow \forall s \in S : m_{1}(s) \leq m_{2}(s)$. (Comparison)
		\item $\left | m \right | = \sum\limits_{s \in S} m(s)$. (Size)
		\item A multiset m is \textbf{infinite} if $\left | m \right | = \infty$. Otherwise m is \textbf{finite}. When $m_{1} \ll= m_{2}$, \textbf{subtraction} is defined as:
		\begin{center}
			$\forall s \in S$ : $(m_{2}$ $--$ $m_{1})(s) = m_{2}(s) - m{1}(s).$
		\end{center}
	\end{enumerate}
	The set of all multisets over S, i.e., the multiset type over S is denoted $S_{MS}$. The
	empty multiset over S is denoted by $\emptyset_{MS}$ and is defined by $\emptyset_{MS}(s) = 0$ for all $s \in S$.
\end{defs}
\subparagraph*{\textnormal{This covers the basic definition of multisets and some of their operations namely addition, scalar multiplication, subtraction, size etc. In this definition there are new symbols defined for addition ($+\!+$), scalar multiplication ($\ast\!\ast$), comparison ($\ll=$) and subtraction ($--$). In further discussions, we would use these notations.}}



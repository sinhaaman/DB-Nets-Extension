\begin{thebibliography}{XX}
\small
\bibitem[\protect\citeauthoryear{Comment}{}]{Comment}
According to FUB-CS standards a chapter containing bibliographic
references should always be included in your dissertation.
It is specified by:
\begin{verbatim}
  \begin{thebibliography}{XX}
    <your list of \bibitems>
  \end{thebibliography}
\end{verbatim}
\bibitem[\protect\citeauthoryear{Comment}{}]{Comment}
If you do not want numbers, then you can you can use, for instance, the very nice and powerful \texttt{natbib} package and have the references listed like the four below (and many options to vary citation in the text), or manually specify something like
\begin{verbatim}
\bibitem[L94]{Lamport}
\end{verbatim}
in your bibliography list to get \verb|[L94]| both in the text and between the square brackets in the bibliography. 
\bibitem[\protect\citeauthoryear{Lamport}{1994}]{Lamport}
Lamport, L. {\em \LaTeX{} User's Guide \& Reference
Manual\/}, Addison-Wesley Publishing Company, Reading, Mass. 1986, 1994.
\bibitem[\protect\citeauthoryear{Poggi and Ruzzi}{2004}]{Poggi04}
Poggi, A., Ruzzi, M. Filling the gap between data federation and data integration. In: di Pula, S.M. (ed.): \emph{Proceedings of the 12th Italian Symposium on Advanced Database Systems}, Cagliari, Italy. 2004. pp270-281.

\bibitem[\protect\citeauthoryear{Pontow and Schubert}{2006}]{Pontow06}
	Pontow, C., Schubert, R. A mathematical analysis of theories of parthood. \emph{Data \& Knowledge Engineering}, 2006, 59:107-138.

\bibitem[\protect\citeauthoryear{Popper}{1996}]{Popper96}
	Popper, K.R. \emph{The myth of the framework -- in defence of science and rationality}. London: Routledge. 1996. 229p.
\end{thebibliography}


